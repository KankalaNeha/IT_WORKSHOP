\documentclass[leqno]{article}
\usepackage{sectsty}
\usepackage{amssymb}
\usepackage{graphicx}
\usepackage{amsmath}
\usepackage{amsthm}
\usepackage{fancyhdr}
\usepackage{hyperref}
\usepackage[paperwidth=216mm,paperheight=279mm,margin=1.5in]{geometry}

\hypersetup{
	colorlinks=true,
	linkcolor=blue,
	urlcolor=blue,
	breaklinks=true,
	pdfborder={0 0 0}, % No border around links
}
\pagestyle{fancy}
\fancyhf{} % Clear header and footer
\fancyhead[R]{\thepage} % Page number on the top right side
\renewcommand{\headrulewidth}{0pt} % Remove header line

\newtheorem{conjecture}{Conjecture}[section]
\newtheorem{theorem}{Theorem}[section]

\title{\textbf{THE 18.821 MATHEMATICS PROJECT LAB REPORT [MATHEMATICS PROJECT]}}
\author{X. BURPS, P. GURPS}
\date{}
\sectionfont{\centering\normalfont\large} % Centering the section title and setting it to normal font
\begin{document}
	\maketitle
	\fontsize{12pt}{14pt}
	ABSTRACT. This is a \LaTeX{} template for 18.821, which you can use for your own reports.
	\vspace{1.5 cm}
	\section{INTRODUCTION}
	\hspace{0.5 cm}
	This brief document shows some examples of the use of \LaTeX and
	indicates some special features of the Math Lab report style.\href{http://stellar.mit.edu/S/course/18/sp13/18.821/}{\underline{The course website}} contains links to several \LaTeX manuals.
	\par{}End the introduction by describing the contents of the paper sec­
	tion by section, and which team member(s) wrote each of them. For
	instance, Section 6 discusses referencing, and is written by P. Gurps.
	\section{\LaTeX EXAMPLES}
	\hspace{0.5 cm}
	Here are some ways of producing mathematical symbols. Some are pre-defined either in \LaTeX or in the
	package which this document loads.For instance sums and integrals, $ \sum_{i=1}^{n} 1=n ,\int^a_b xdx =\frac{n^2}{2}.$ We’ve defined a few other symbols at the start of the document, for
	instance $\mathbb{N,Q,Z,R}$ You can make marginal notes for yourself or your
	co-authors like this: \hspace{2 cm}	Unfinished here?\par{}
	If you want to typeset equations, there are many choices, with or
	without numbering:
	\[	\int^0_1 xdx= \frac{1}{2}, \]
	\raggedright 
	or  
	\[	\sum_{i=1}^{\infty}i=\frac{-1}{12} \]
	\raggedright
	or
	\[  1-1+1-...=\frac{1}{2}.\]
	\newpage
	\section*{X. BURPS, P. GURPS}
	\begin{figure}[h]
		\centering
		\includegraphics[width=0.6\linewidth]{pic}
		\caption{My first pdf figure.}
		\label{fig:pic}
	\end{figure}
	If you want a number for an equation, do it like this:
	\begin{align}
		\lim_{n \to \infty} \sum_{k=1}^{n} \frac{1}{k^2}=\frac{\pi}{6}.
	\end{align}
	This can then be referred to as (1), which is much easier than keeping
	track of numbers by hand. To group several equations, aligning on the
	= sign, do it like this:
	\begin{align*}
		x_1 + 2x_2 + 3x_3 &= 7 \\
		y &= mx + c \\
		&= 4x - 9.
	\end{align*}
	You can easily embed hyperlinks into the output .pdf document:
	\href{http://stellar.mit.edu/S/course/18/sp13/18.821/}{\underline{click here}} for example.
	\section{IMAGES}
	\hspace{0.5 cm}Figure 1 is an example of a .pdf image put into a floating environ­
	ment, which means LaTeX will draw it wherever there’s enough space
	left in your manuscript. Look at the .tex original to see how to insert
	a figure like this.
	\section{THEOREMS AND SUCH}
	\hspace{0.5 cm}An example of a “conjecture environment” is given below, in Con­jecture 4.1. Theorems, lemmas, propositions, definitions, and such all
	use the same command with the appropriate name changed. In fact,if you look at the top of this .tex file, you can see where we’ve defined
	these environments.\\
	\newpage
	\section*{THE 18.821 REPORT}
	\begin{conjecture}[Vaught’s Conjecture]
		Let $T$ be a countable complete theory. If $T$ has fewer than $2^{\aleph_0}$ many countable models (up to isomorphism), then it has countably many countable models.
	\end{conjecture}
	
	\begin{theorem}
		When it rains it pours.
	\end{theorem}
	\begin{proof}
		Well, yes.
	\end{proof}
	\section{FILETYPES USED BY LATEX}
	\hspace{0.5 cm}You will write your text as a \texttt{.tex} file using any text editor (though WYSIWYG editors are troublesome). Traditionally one then runs \LaTeX{} and obtains a \texttt{.dvi} file, which can be viewed on the screen using a dvi viewer. To include images, and then prepare the file for printing or submission, one typically translates the \texttt{.dvi} into either \texttt{.ps} (Postscript) or \texttt{.pdf} (Adobe PDF).
	\par{}\hspace{0.5 cm}Your report will be submitted as a \texttt{.pdf} document. The \texttt{pdflatex} command produces a \texttt{.pdf} file directly from a \texttt{.tex} file. This command works well with included \texttt{.pdf} files, but does not handle \texttt{.eps} files. An \texttt{.eps} file can be converted to a \texttt{.pdf} file by viewing it and saving as a \texttt{.pdf} file, or by \texttt{ps2pdf filename.eps}, which produces \texttt{filename.pdf}.Under MikTeX with WinEdt, all necessary commands will appear under “Accessories” in the WinEdt menu.
	\par{}\hspace{0.5 cm} Finally, Matlab can be made to produce \texttt{.eps} files by typing\\
	\hspace{3 cm} \texttt{print -deps filename}\\ at the prompt.
	\section{QUOTING SOURCES}
	\hspace{0.5 cm}In your work, keep notes of the literature you’ve used, including websites. Cite the references you use; failure to do so constitutes plagiarism. Every bibliography item should be referenced somewhere in the paper. Quote as precisely as possible:~\cite{ref1}[pages 76--78] rather than~\cite{ref1}.~\cite{ref2} was a useful background reference, too.
  
  \cite{gurps2008,burps2008}
  \bibliographystyle{plain}
  \bibliography{ref12.bib}
  
	\appendix
	\section*{APPENDIX}
	Appendices are useful for putting in code or data.\\ 
	\newpage
	MIT OpenCourseWare\\
	\href{http://ocw.mit.edu}{\underline{http://ocw.mit.edu}}\\
	\vspace{2 cm}
	18.821 Project Laboratory in Mathematics
	\newline
	Spring 2013

	\vspace{2 cm}
	For information about citing these materials or our Terms of Use, visit: 
	\href{http://ocw.mit.edu/terms}{\underline{http://ocw.mit.edu/terms}}
\end{document}
